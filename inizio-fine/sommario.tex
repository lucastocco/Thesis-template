% !TEX encoding = UTF-8
% !TEX TS-program = pdflatex
% !TEX root = ../tesi.tex

%**************************************************************
% Sommario
%**************************************************************
\cleardoublepage
\phantomsection
\pdfbookmark{Sommario}{Sommario}
\begingroup
\let\clearpage\relax
\let\cleardoublepage\relax
\let\cleardoublepage\relax

\chapter*{Sommario}
Il presente documento descrive il lavoro da me svolto durante il periodo di stage presso l'azienda \AD{} di Padova.\\
Ho svolto lo stage al termine del percorso di studi della Laurea Triennale, sotto la supervisione del tutor aziendale Antonio Fornari, per una durata complessiva di 304 ore.\\
L'obiettivo dello stage era di partecipare alla fase di sviluppo di due distinti progetti negli ambiti dell'Industria 4.0 e della \textit{digital transformation}.
\begin{itemize}
    \item \textbf{{\hyperref[cap:introduzione]{Il primo capitolo}}} descrive il contesto aziendale nel quale ho svolto il tirocinio: nello specifico espone la storia, prodotti e modello di business.
    
    \item \textbf{{\hyperref[cap:obiettivi-stage]{Il secondo capitolo}}} riporta la proposta e gli obiettivi di stage, oltre alle motivazioni che mi hanno spinto a questa scelta.
    
    \item \textbf{{\hyperref[cap:descrizione-stage]{Il terzo capitolo}}} descrive nel dettaglio le tecnologie, le attività e le scelte progettuali adottate durante il tirocinio.
    
    \item \textbf{{\hyperref[cap:analisi-requisiti]{Il quarto capitolo}}} riporta una valutazione retrospettiva dello stage, tracciando un bilancio degli obiettivi raggiunti e considerando l'eventuale crescita professionale.
\end{itemize}
Riguardo la stesura del testo, relativamente al documento sono state adottate le seguenti convenzioni tipografiche:
\begin{itemize}
	\item gli acronimi e le abbreviazioni sono definiti nella lista degli acronimi, mentre i termini ambigui o di uso non comune menzionati sono esplicati nel glossario, entrambi situati alla fine del presente documento;
	\item per la prima occorrenza dei termini riportati nel glossario viene utilizzata la seguente nomenclatura: \markg{parola};
	\item i termini in lingua straniera o facenti parti del gergo tecnico sono segnalati con il carattere \textit{corsivo};
	\item i termini che necessitano un certo rilievo nel contesto, sono evidenziati con il carattere \textbf{grassetto}.
\end{itemize}
\endgroup			

\vfill

