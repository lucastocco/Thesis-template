% !TEX encoding = UTF-8
% !TEX TS-program = pdflatex
% !TEX root = ../tesi.tex
%**************************************************************
\cleardoublepage
\chapter{Valutazione retrospettiva}
\label{cap:valutazione-retrospettiva}
%**************************************************************
\section{Bilancio degli obiettivi raggiunti}
- tabella che indica gli obiettivi soddisfatti e non soddisfatti \\
- spiegazione degli obiettivi non soddisfatti\\
- bilancio degli obiettivi personali
%**************************************************************
\section{Conoscenze e competenze acquisite}
- competenze tecniche (linguaggi di programmazione, utilizzo dei tool di lavoro)\\
- competenze progettuali (punti di forza e limiti di alcune soluzioni rispetto ad altre)\\
- esperienza professionale\\

\section{Valutazione personale}
- difficile approccio con Angular in quanto javascript/typescript è stato poco trattato durante il corso accademico. Fortunatamente, l'architettura MVC mi era già nota, e questo ha facilitato un po' la comprensione del suo funzionamento\\
- il corso accademico dovrebbe fornire agli studenti una base sugli strumenti di supporto allo sviluppo più comuni (versionamento, framework di testing, CI/CD)