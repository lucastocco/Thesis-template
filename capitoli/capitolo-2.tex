% !TEX encoding = UTF-8
% !TEX TS-program = pdflatex
% !TEX root = ../tesi.tex
\newpage
%**************************************************************
\chapter{Obiettivi dello stage}
\label{cap:obiettivi-stage}
%\intro{Brevissima introduzione al capitolo}\\
%%%%%%%%%%%%%%%%%%%%%%%%%%%%%%%%%%%%%%%%%%%%%%%%%%%%%%%%%%%%%%%%%%%%%%%%%%%%%%%%%%%%%%%%%%%
\section{Lo stage nella strategia aziendale}
%Importanza dello stage in AzzurroDigitale
%%%%%%%%%%%%%%%%%%%%%%%%%%%%%%%%%%%%%%%%%%%%%%%%%%%%%%%%%%%%%%%%%%%%%%%%%%%%%%%%%%%%%%%%%%%
\subsection{Vantaggi Aziendali}
%Vantaggi per l'azienda: \\
%- partecipazione a stageIT consente all'azienda di entrare in contatto con i laureandi\\
%- formazione di personale giovane e selezionato\\
%- nuove idee e punti di vista \\
%%%%%%%%%%%%%%%%%%%%%%%%%%%%%%%%%%%%%%%%%%%%%%%%%%%%%%%%%%%%%%%%%%%%%%%%%%%%%%%%%%%%%%%%%%%
\subsection{Presentazione dei progetti}
%descrizione generale dei progetti: \\
%- le idee che stanno dietro a questi\\
%- peculiarità\\
%- differenze con la concorrenza\\
%%%%%%%%%%%%%%%%%%%%%%%%%%%%%%%%%%%%%%%%%%%%%%%%%%%%%%%%%%%%%%%%%%%%%%%%%%%%%%%%%%%%%%%%%%%
\subsection{Aspettative aziendali}
%definizione e classificazione degli obiettivi da raggiungere

Presentati i progetti, si è passati alla fase di definizione dei traguardi da raggiungere durante lo stage.
Il team di sviluppo ha deciso di suddividere questi obiettivi in due categorie: gli \textbf{obiettivi minimi} si riferiscono a dei compiti il cui completamento risulta essere indispensabile per l'avanzamento del progetto, gli \textbf{obiettivi opzionali} invece, fanno riferimento a delle caratteristiche del prodotto di importanza minore e quindi il loro soddisfacimento non è stato considerato obbligatorio.
Di seguito, il riepilogo degli obiettivi che mi sono stati assegnati, ripartiti tra i vari progetti.

\subsubsection*{DigitalSnapshots}
\paragraph*{Obiettivi Minimi}
\begin{itemize}
\item Ristrutturazione della base di dati MySQL e nella parte Model di CakePHP;
\item Realizzazione moduli API e logica;
\item Creazione interfaccia utente del cruscotto delle analisi; 
\item Stesura della documentazione su quanto realizzato.
\end{itemize}

\paragraph*{Obiettivi Opzionali}
\begin{itemize}
\item Realizzazione di un modulo "wizard" per la configurazione rapida del prodotto;
\item Implementazione di una gerarchia di utenti, con livelli di privilegi differenti.
\end{itemize}

\subsubsection*{AWMS}
\paragraph*{Obiettivi Minimi}
\begin{itemize}
\item Implementazione dei dati necessari nel database PostgreSQL e nella parte Model di CakePHP;
\item Realizzazione moduli API e logica;
\item Creazione interfaccia utente per la selezione della tipologia e delle opzioni di stampa; 
\item Stesura della documentazione su quanto realizzato.
\end{itemize}

\paragraph*{Obiettivi Opzionali}
\begin{itemize}
\item Realizzazione di un modulo per la configurazione dei ruoli degli utenti.
\end{itemize}

\section{Vincoli}
\subsection{Vincoli temporali}
%ore di lavoro complessive, scadenze sprint e scadenze di progetto

La durata complessiva dell'attività di stage è stata di 304 ore, distribuite, in accordo con il tutor aziendale, nell'arco di 8 settimane, ognuna delle quali aveva un monte orario di circa 40 ore. \\
L'orario lavorativo stabilito corrisponde all'orario di lavoro aziendale: dal Lunedì al Venerdi, dalle ore 9:00 alle 18:00.\\
Oltre a questo vincolo del monte orario, mi sono trovato a toccare con mano la pianificazione Scrum: all'inizio di ogni Sprint, infatti, ho pianificato le attività da portare a termine entro le successive due settimane, durata di ogni singolo Sprint.//
Il progetto \textbf{DigitalSnapshots} inoltre, aveva una scadenza di consegna al cliente che combaciava con la fine del secondo Sprint, per cui il ritmo lavorativo è stato abbastanza elevato per non eccedere tale data.

\subsection{Vincoli metodologici}
%- monday meeting\\
%- interazione diretta con il cliente\\
%- scrum\\
%- Vincoli metodologici personali
%%%%%%%%%%%%%%%%%%%%%%%%%%%%%%%%%%%%%%%%%%%%%%%%%%%%%%%%%%%%%%%%%%%%%%%%%%%%%%%%%%%%%%%%%%%
\subsection{Vincoli tecnologici}
%stack tecnologico definito in avvio di progetto 

Per la realizzazione dei due progetti di stage, ho utilizzato due \textit{stack} tecnologici molto simili, ma per certi aspetti radicalmente differenti.\\
Il progetto \textbf{AWMS} si basa su tecnologie ampiamente utilizzate in azienda: 
\begin{itemize}
\item \textbf{PostgreSQL:} database relazionale, compatibile con il paradigma SQL, che consente all'utente di immagazinare dati anche in formato Json, rendendo di fatto il contenuto delle tabelle molto più dinamico;
\item \textbf{CakePHP:} \textit{framework} che consente lo sviluppo rapido di applicazioni web con architettura MVC. Grazie alla sua funzione di \textit{ORM}, permette di effettuare operazioni sulle tabelle trattando quest'ultime come oggetti derivanti dal paradigma \textit{OOP};
\item \textbf{Angular2+:} \textit{framework} per lo sviluppo di applicazioni web \textit{single-page}.
\end{itemize}
\textbf{DigitalSnapshots} invece si differenzia da AWMS in quanto sostituisce \textbf{MySQL} a \textit{PostgreSQL} e \textbf{AngularJS} ad \textit{Angular2+}.\\
Questa differenza di comparto tecnico è giustificata dal fatto che, mentre AWMS verrà seguito da \AD{} (nello specifico, da \textbf{I4.0Saas} dal 2020) per tutto il suo ciclo di vita, \textbf{DigitalSnapshots} avrà un destino differente: le fasi di progettazione e sviluppo sono a carico di \AD{}, mentre la fase di manutenzione del codice sarà eseguita dal reparto tecnico di Electrolux, committente del progetto.\\
Dai vincoli metodologici personali, deriva l'utilizzo dei \textit{framework} \textit{Jenkins}, per l'implementazione della \textit{continous integration}, e \textit{PHPUnit} e \textit{Jasmine} per l'esecuzione automatica dei test d'unità, rispettivamente per il linguaggio di programmazione PHP e Typescript.
\section{Aspettative personali}
%- come ho conosciuto AD\\
%- perchè AD?\\
%- aspettative sul lavorare in una startup, imparare il way of working\\
%%%%%%%%%%%%%%%%%%%%%%%%%%%%%%%%%%%%%%%%%%%%%%%%%%%%%%%%%%%%%%%%%%%%%%%%%%%%%%%%%%%%%%%%%%%