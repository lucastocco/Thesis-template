% !TEX encoding = UTF-8
% !TEX TS-program = pdflatex
% !TEX root = ../tesi.tex
\newpage
%**************************************************************
\chapter{Descrizione dello stage}
\label{cap:descrizione-stage}

\section{Introduzione ai progetti}
breve introduzione
\subsection{DigitalSnapshot}
- descrizione\\
- architettura\\
- obiettivi\\
\subsection{AWMS}
- descrizione\\
- architettura\\
- obiettivi\\
%**************************************************************
\section{\textit{Stack} tecnologico}
\subsection{CakePHP}
framework che facilita la creazione di applicazioni web
\subsection{PHPUnit}
framework per testare codice php
\subsection{MySQL e PostgreSQL}
descrizione dei database utilizzati
\subsection{AngularJS e Angular2+}
framework per realizzazione della parte front-end
\subsection{Jenkins}
framework per continuous integration
\subsection{Redis e Docker}
descrizione applicazioni ed utilizzo
%**************************************************************

\section{Integrazione delle tecnologie utilizzate}
descrizione di come sono stati integrati tra loro i vari framework utilizzati
%**************************************************************

\section{Pianificazione}
pianificazione in base alle scadenze di progetto, agli sprint programmati e agli impegni accademici personali
%**************************************************************

\section{Analisi dei requisiti}
- descrizione di come sono stati stilati i requisiti\\
- classificazione dei requisiti\\
- definizione dei requisiti\\
%**************************************************************

\section{Progettazione}
descrizione della sezione, se necessaria
\subsection{Progettazione basi di dati}
- importanza di una buona progettazione del db\\
- modifiche apportate alle basi di dati\\

\subsection{Progettazione API}
- architettura REST, quindi importanza delle API\\
- progettazione e descrizione API più significative\\

\subsection{Progettazione interfacce utente}
- da mockup a GUI\\
- limitazioni applicative (per natura della webapp => poco responsive e poca accessibilità)\\
%**************************************************************

\section{Sviluppo}
\subsection{Ambiente di sviluppo}
- tool utilizzati (phpstorm, postman, DevTools di chrome, jenkins)\\
- modalità di sviluppo (norme di stesura del codice, TDD, documentazione del codice)\\
- utilizzo di git e gitflow per lo sviluppo in team\\
\subsection{Implementazione basi di dati}
- creazione tabelle \\
- Model, Table, Entity (cakephp)\\
- Migrations\\
\subsection{Implementazione moduli di stampa}
- stampa su file .xlsx e/o .pdf\\
- front-end => dialog con scelta di opzioni di stampa\\
- back-end => design pattern applicati e principi SOLID\\
\subsection{Implementazione cruscotti delle analisi}
- importanza dei cruscotti di analisi
- implementazione libreria chart.js
%**************************************************************

\section{Verifica}
\subsection{Analisi statica}
- front-end => ESLint\\
- back-end => Parallel Lint, CodeSniffer e PHPStan\\
\subsection{Analisi dinamica}
- creazione dei test prima della codifica (TDD)\\
- classificazione dei test\\
- esecuzione automatica dei test d'unità e di integrazione. Test di sistema effettuati manualmente \\
- framework utilizzati: phpUnit per back-end, karma/jasmine per front-end\\
%**************************************************************

\section{Validazione}
bilancio dei requisiti soddisfatti\\
%**************************************************************